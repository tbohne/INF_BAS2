\documentclass[12pt, a4paper]{article}
\usepackage{a4wide}

\usepackage[utf8]{inputenc}
\usepackage[ngerman]{babel}
\usepackage[T1]{fontenc}
\usepackage{palatino} %font

\usepackage{graphicx}
\usepackage{caption}
\usepackage{subcaption} %für subfigures
\usepackage{url}
\usepackage{tocloft}
\usepackage{acronym}

\usepackage[babel,german=quotes,threshold=3]{csquotes} 

\usepackage{lipsum}

%Pfad für Grafiken
\graphicspath{{img/}}

%Styleregeln
\widowpenalty10000 % Vermeidet einzelne Zeilen eines Absatzes zu Beginn einer Seite
\clubpenalty10000 % Vermeidet einzelne Zeilen eines Absatzes am Ende einer Seite
\addtocontents{toc}{\protect\sloppy}
\setcounter{tocdepth}{3}

\begin{document}

%deaktiviere Seitenzahlen
\pagenumbering{gobble}

%Titelseite
\begin{titlepage}
\centering
\thispagestyle{empty}
\begin{center}
\includegraphics[width=0.9\textwidth]{uos.pdf}
\end{center}
\LARGE{\textsc{Institut für Informatik\\Arbeitsgruppe Verteilte Systeme}}
\vfill
\LARGE{\emph{Seminar}}\\
\LARGE{\emph{Mobility and Traffic in Computer Networks}}\\
\vspace{8mm}
\huge{\textbf{{\fontfamily{ppl}\selectfont
Analyzing Mobility-Traffic Correlations in Large WLAN Traces}}}\\
\vspace{9mm}
\LARGE{Tim Bohne}\\
\vspace{0.2cm}
%ACHTUNG: !!!Matrikelnummer nur für die Abgabeversion, NICHT mit ins Wiki hochladen!!!
% \normalsize{Matrikelnummer}\\
\vspace{4cm}
\large{Sommersemester 2019}\\
\vspace{0.2cm}
\large{\today}
\vfill
\end{titlepage}
\newpage

%Inhaltsverzeichnis
\tableofcontents
\newpage

\pagestyle{plain}
\pagenumbering{arabic} %Starte Seitennummerierung

\section{Einleitung}

Die stetig wachsende Zahl mobiler Geräte führt zu enormen Herausforderungen bei der Entwicklung
und Planung der dafür erforderlichen Infrastruktur. Beim Konzipieren neuer Netzwerke ist es von zentraler
Bedeutung, diese in möglichst realistischer Weise simulieren zu können, um sinnvolle Designentscheidungen
treffen zu können. Dabei spielen zwei Bereiche eine essenzielle Rolle, die Bewegung der Nutzer bzw. Geräte
sowie der Datenverkehr innerhalb des Netzwerks.\newline\newline
Diese Ausarbeitung befasst sich mit dem Paper \textquote{Flutes vs. Cellos: Analyzing Mobility-Traffic
Correlations in Large WLAN Traces} \cite{Alipour2018} aus dem Jahr 2018, in welchem Korrelationen
zwischen diesen Bereichen untersucht werden. Darüber hinaus...\newline\newline

Es thematisiert Storage Loading Probleme mit Stacking Constraints.
Ziel des Papers ist es, erste theoretische Ergebnisse für diese
Klasse von Problemen zu liefern. Zusätzlich zu Ergebnissen der NP-Schwere, werden
Polynomialzeit-Algorithmen für einige der Probleme vorgestellt, die als Bausteine in
heuristischen Ansätzen zur Lösung allgemeinerer Probleme genutzt werden können.
Das Paper ist dadurch motiviert, dass bisher kaum Komplexitätsergebnisse für Storage
Loading Probleme vorliegen [10], obwohl diese auch aus praktischer Perspektive eine
durchaus relevante Klasse von Problemen darstellen.

Zunächst werden in Kapitel 2 Storage Loading Probleme im Allgemeinen eingeführt
und anschließend in Kapitel 3 formal definiert. Darauffolgend wird in Kapitel 4 auf die
„Stacking Constraints“ eingegangen, die in Storage Loading Problemen eine zentrale
Rolle spielen. In Kapitel 5 wird die simpelste Variante eines Storage Loading Problems
skizziert und in Kapitel 6 durch mögliche Zielfunktionen ergänzt. Im Anschluss geht
es dann in Kapitel 7 darum, die im Paper hergeleiteten Theoreme und zugehörigen
Beweise vorzustellen. Kapitel 8 enthält einige abschließende Bemerkungen.

\pagebreak

% \cite{aaronson}\cite{flocchini}\cite{upper}

\section{Grundlagen}

\subsection{Mobilität}
\label{sec:mobility}

Die Performance eines kabellosen Kommunikationssystems hängt unter anderem von der Bewegung
der mobilen Knoten innerhalb dieses Systems ab. Dementsprechend ist es erstrebenswert,
solche Bewegungsmuster zu emulieren, um möglichst realistische Simulationen durchführen zu können.
Ein Mobilitätsmodell beschreibt das Bewegungsmuster mobiler Benutzer bzw. Geräte,
also wie sich deren Standort, Geschwindigkeit und Beschleunigung im Laufe der Zeit ändern.
Es geht im Wesehntlichen darum, künstliche Bewegungsmuster zu generieren, die zu jeder Zeit reproduzierbar sind.
\newline\newline
Dabei wird zwischen zwei Modellen unterschieden, den \textquote{Entity-Mobility-Models},
also Modelle, bei denen die Bewegung einzelner Entitäten modelliert wird und den \textquote{Group-Mobility-Models},
bei welchen es darum geht, die Bewegung einer Gruppe von Nutzern bzw. Geräten zu modellieren.
Bei dieser Unterscheidung geht es im Wesentlichen darum, ob die Bewegung der einzelnen Entitäten voneinander abhängt oder nicht.
Dies ist nur beim Group-Mobility-Model der Fall.
\newline\newline
Um neue Protokolle für Netzwerke zu simulieren, ist es zwingend erforderlich, ein Mobilitätsmodell zu verwenden,
welches die mobilen Knoten geeignet repräsentiert, sodass die Bewegung tatsächlicher Nutzer möglichst realistisch abgebildet wird.
\newline\newline
Neben der Unterscheidung zwischen Entity- und Group-Mobility-Models wird auch zwischen zwei Arten der Datenbasis unterschieden.
Es gibt zum Einen die Bewegungsmodelle, die auf Trace-Daten basieren, d.h. auf Beobachtungen der Bewegung von Nutzern
in tatsächlich existierenden Systemen. Diese liefern bei einer ausreichend großen Gruppe von Mobilnutzern und einer ausreichend
langen Beobachtungsphase sehr akurate Informationen, setzen allerdings voraus, dass ein solches Netzwerk bereits existiert.
Die zweite Variante entpsricht der synthetischen Generierung von Bewegungsmustern für Netzwerkumgebungen, für die noch keine
Traces vorliegen. Dabei geht es dann darum, die mobilen Knoten möglichst realistisch abzubilden, wobei es im Wensentlichen vier
Arten von Ansätzen gibt, dies zu tun: Rein zufällig, basierend auf zeitlichen oder räumlichen Abhängigkeiten und basierend
auf geographischen Restriktionen. Basierend auf diesen vier Ansätzen existieren eine Vielzahl von Mobilitätsmodellen,
auf welche ggf. im Verlauf der Ausarbeitung an geeigneter Stelle Bezug genommen wird.

\pagebreak

\subsection{Datenverkehr}
\label{sec:traffic}

Zu einer sinnvollen Simulation der Performance eines Kommunikationssystems gehört auch die Simulation des Datenverkehrs
innerhalb dieses Systems. Wie bereits bei der Bewegungsmodellierung kann auch hier von tatsächlichen Beobachtungen in
existierenden Netzen ausgegangen werden, wenn solche vorliegen.

Einfacher Datenverkehr besteht aus der einfachen Ankunft diskreter Entitäten, z.B. Paketen.
Es gibt allerdings auch Datenverkehr, bei welchem mehrere Einheiten eintreffen, sogenannte \textquote{Batch Arrivals}.
Im einfachsten Fall wird in einer Simulation einfach eine zufällig generierte Sequenz von Zwischenankunftszeiten generiert.
Neben Ankunftszeiten und Batch-Größen möchte man evtl. auch die Last modellieren.

Ein einfaches Modell des Datenverkehrs ist \textsc{CBR} (Constant Bitrate), bei welchem eine Anzahl von Paketankünften
pro Zeiteinheit definiert wird. Die Paketgröße ist dabei ebenfalls konstant. Dieses Modell ist sehr simpel und Dementsprechend
leicht zu implementieren, aber auch nicht besonders realistisch. Wenngleich VoIP und Video-Übertragung manchmal
die Gestalt von \textsc{CBR}-Datenverkehr aufweist.

Man kann verschiedene Parameter variieren, um realistischen Datenverkehr zu erzeugen. Die Zeit kann z.B. diskret oder
kontinuierlich sein. Es können verschiedene Ströme des Datenverkehrs multiplext werden, um einen realistischen Mix
des Datenverkehrs zu erhalten, wie er z.B. durch Sprachverbindungen, Video-Übertragungen, Datei-Transport etc. entsteht.

Neben \textsc{CBR} gibt es auch \textsc{VBR} (Variable Bitrate), welcher z.B. beim Videostreaming auftritt,
bei welchem kodierte Frames eine variable, zufällige Größe haben.

Eine weitere wichtige Eigenschaft des Datenverkehrs ist die sogenannte \textquote{Burstiness}, also die Stoßhaftigkeit.
Traffic entsteht in Schüben, z.B. für komprimierte Videos und Datentransfers. Die Ankünfte der Pakete bilden visuelle
Cluster. Ein sehr großer Teil des Datenverkehrs in Breitbandnetzen wird von dieser Art Traffic dominiert.

Wie bereits bei der Bewegungsmodellierung gibt es auch hier eine Vielzahl verschiedener Modelle, die in bestimmten Szenarien
geeignet sein können, um realistischen Datenverkehr zu modellieren.

\pagebreak

\section{Korrelationen zwischen Mobilität und Datenverkehr}

Wie in den Abschnitten \ref{sec:mobility} und \ref{sec:traffic} deutlich wurde, hängt die Performanz eines mobilen
Netzwerks erheblich von den Mustern der Mobilität und des Datenverkehrs ab.
\newline\newline
Aktuelle Modelle betrachten entweder die Mobilität oder den Datenverkehr, erfassen jedoch nicht das Zusammenspiel beider Faktoren.
\cite{Alipour2018} Außerdem basieren viele trace-basierte Mobilitätsmodelle auf Datensätzen aus der Prä-Smartphone-Ära,
was die heutige Relevanz dieser Datensätze in Frage stellt. Die Autoren beschäftigen sich mit Mobilitäts- und Datenverkehrmodellen
für Laptops und Smartphones.
\newline\newline
Die Modelle sind datengetrieben, es werden $30$\textsc{TB} große Datensätze von $300 k$ Geräten betrachtet, welche auf einem Campusgelände
betrieben wurden. Zur Analyse dieser Daten wurde das Framework FLAMeS (\textbf{F}ramework for \textbf{L}arge-scale \textbf{A}nalysis of \textbf{M}obil\textbf{e} \textbf{S}ocieties) entwickelt. Das Ziel ist im Wesentlichen, einen ersten Schritt in Richtung integrierter
Mobilitäts-Datenverkehr-Modelle zu ermöglichen, um in der Zukunft realistischere Test-Szenarien und Benchmarks zu entwickeln.
\newline\newline
Die beiden Faktoren der Mobilität und des Datenverkehrs beeinflussen sich durchaus gegenseitig. Eine Person kann langsamer gehen,
wenn diese eine Nachricht erhält. Ebenso kann die Netzwerkaktivität von der Mobilität und Position abhängen. Stationäre Nutzer
konsumieren und produzieren i.d.R. mehr Daten als jene, die sich bewegen. Außerdem verwenden Personen an unterschiedlichen
Orten undterschiedliche Anwendungen und Services. Trotz dieses Zusammenhangs wurden Modelle zur Mobilität und zum Datenverkehr
bisher typischerweise isoliert voneinander betrachtet.
\newline\newline
Um diesen Zusammenhang genauer zu untersuchen, wird zwischen stationären Geräten (Laptops) und mobilen Geräten (Smartphones)
unterschieden. Das bereits erwähnte Framework FLAMeS wurde entwickelt, um stationäre und mobile Geräte zu differenzieren
und in Bezug auf ihren Datenverkehr und ihre Mobilität zu untersuchen.
\newline\newline
Im Kern wurden dafür drei Fragestellungen untersucht:\newline
\begin{itemize}
    \item Wie unterscheiden sich Mobilität und Datenverkehr zwischen den unterschiedlichen Geräteklassen, Zeit und Ort?
    \item Was sind die Beziehungen dieser Charakteristiken?
    \item Sollten neue Modelle entwickelt werden, die diese Unterschiede berücksichtigen? Wenn ja, wie?
\end{itemize}

\pagebreak

\section{Schluss}

%Bibliographie
\addcontentsline{toc}{section}{Literatur}
\bibliographystyle{IEEEtranSA}
%\bibliographystyle{acm}
\bibliography{sources.bib}

\end{document}
