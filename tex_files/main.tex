\documentclass[12pt, a4paper]{article}
\usepackage{a4wide}

\usepackage[utf8]{inputenc}
\usepackage[ngerman]{babel}
\usepackage[T1]{fontenc}
\usepackage{palatino} %font

\usepackage{graphicx}
\usepackage{caption}
\usepackage{subcaption} %für subfigures
\usepackage{url}
\usepackage{tocloft}
\usepackage{acronym}
\usepackage{float}

\usepackage[babel,german=quotes,threshold=3]{csquotes} 

\usepackage{lipsum}
\usepackage{hyperref}  %hyperref still needs to be put at the end!

%Pfad für Grafiken
\graphicspath{{img/}}

%Styleregeln
\widowpenalty10000 % Vermeidet einzelne Zeilen eines Absatzes zu Beginn einer Seite
\clubpenalty10000 % Vermeidet einzelne Zeilen eines Absatzes am Ende einer Seite
\addtocontents{toc}{\protect\sloppy}
\setcounter{tocdepth}{3}

\begin{document}

%deaktiviere Seitenzahlen
\pagenumbering{gobble}

%Titelseite
\begin{titlepage}
\centering
\thispagestyle{empty}
\begin{center}
\includegraphics[width=0.9\textwidth]{uos.pdf}
\end{center}
\LARGE{\textsc{Institut für Informatik\\Arbeitsgruppe Verteilte Systeme}}
\vfill
\LARGE{\emph{Seminar}}\\
\LARGE{\emph{Mobility and Traffic in Computer Networks}}\\
\vspace{8mm}
\huge{\textbf{{\fontfamily{ppl}\selectfont
Analyzing Mobility-Traffic Correlations in Large WLAN Traces}}}\\
\vspace{9mm}
\LARGE{Tim Bohne}\\
\vspace{0.2cm}
%ACHTUNG: !!!Matrikelnummer nur für die Abgabeversion, NICHT mit ins Wiki hochladen!!!
% \normalsize{Matrikelnummer}\\
\vspace{4cm}
\large{Sommersemester 2019}\\
\vspace{0.2cm}
\large{\today}
\vfill
\end{titlepage}
\newpage

%Inhaltsverzeichnis
\tableofcontents
\newpage

\pagestyle{plain}
\pagenumbering{arabic} %Starte Seitennummerierung

\section{Einleitung}

Die stetig wachsende Zahl mobiler Geräte führt zu enormen Herausforderungen bei der Entwicklung
und Planung der dafür erforderlichen Infrastruktur. Beim Konzipieren neuer Netzwerke ist es von zentraler
Bedeutung, diese in möglichst realistischer Weise simulieren zu können, um sinnvolle Designentscheidungen
treffen zu können. Dabei spielen zwei Bereiche eine essenzielle Rolle, die Bewegung der Nutzer bzw. Geräte
sowie der Datenverkehr innerhalb des Netzwerks.\newline\newline
Diese Ausarbeitung befasst sich mit dem Paper \textquote{Flutes vs. Cellos: Analyzing Mobility-Traffic
Correlations in Large WLAN Traces} \cite{Alipour2018} aus dem Jahr 2018, in welchem Korrelationen
zwischen diesen Bereichen untersucht werden. Darüber hinaus...\newline\newline

Es thematisiert Storage Loading Probleme mit Stacking Constraints.
\newline\newline
Ziel des Papers ist es, die Unterschiede zwischen Laptops und Smartphones in Bezug auf Datenverkehr und Mobilität
zu untersuchen und auch grundsätzlich die Interaktion zwischen Mobilität und Traffic.

Zunächst werden in Kapitel 2 Storage Loading Probleme im Allgemeinen eingeführt
und anschließend in Kapitel 3 formal definiert. Darauffolgend wird in Kapitel 4 auf die
„Stacking Constraints“ eingegangen, die in Storage Loading Problemen eine zentrale
Rolle spielen. In Kapitel 5 wird die simpelste Variante eines Storage Loading Problems
skizziert und in Kapitel 6 durch mögliche Zielfunktionen ergänzt. Im Anschluss geht
es dann in Kapitel 7 darum, die im Paper hergeleiteten Theoreme und zugehörigen
Beweise vorzustellen. Kapitel 8 enthält einige abschließende Bemerkungen.

\pagebreak

% \cite{aaronson}\cite{flocchini}\cite{upper}

\section{Grundlagen}

In diesem Kapitel werden wichtige Konzepte eingeführt, welche im Verlauf der Arbeit eine zentrale Rolle spielen.
In Abschnitt \ref{sec:mobility} geht es um Mobilität in Rechnernetzen und Abschnitt \ref{sec:traffic} handelt
vom Datenverkehr in ebensolchen.

\subsection{Mobilität}
\label{sec:mobility}

Die Performanz eines kabellosen Kommunikationsnetzwerks hängt unter anderem von der Bewegung 
der Nutzer bzw. Geräte innerhalb dieses Netzwerks ab. Dementsprechend ist es erstrebenswert,
solche Bewegungsmuster z.B. bei der Planung neuer Netze zu emulieren, um möglichst realistische Simulationen
durchführen zu können. Ein Mobilitätsmodell beschreibt das Bewegungsmuster mobiler Benutzer bzw. Geräte,
also die Art, in welcher sich deren Standort, Geschwindigkeit und Beschleunigung im Laufe der Zeit ändern. \cite{Camp2002}
Im Wesehntlichen geht es darum, künstliche Bewegungsmuster zu generieren, die die Realität in solch einem Netzwerk
möglichst genau abbilden.
\newline\newline
Dabei wird zwischen zwei grundsätzlichen Modellen unterschieden, den \textquote{Entity-Mobility-Models},
also Modellen, bei denen die Bewegung einzelner Entitäten modelliert wird und den \textquote{Group-Mobility-Models},
bei welchen es darum geht, die Bewegung einer Gruppe von Nutzern bzw. Geräten zu modellieren,
in welcher die Bewegung der einzelnen Entitäten voneinander abhängt.
\newline\newline
Neben der Unterscheidung zwischen Entity- und Group-Mobility-Modellen, wird außerdem zwischen zwei Arten der Datenbasis unterschieden,
welche der Simulation zugrunde liegt. Es gibt zum einen die Bewegungsmodelle, die auf Trace-Daten basieren, d.h. auf Beobachtungen
der Bewegung von Nutzern in tatsächlich existierenden Systemen. Diese liefern bei einer ausreichend großen Gruppe von Nutzern
und einer ausreichend langen Beobachtungsphase sehr akurate Informationen, setzen allerdings voraus, dass ein solches Netzwerk
bereits existiert. \cite{Camp2002} Zum anderen existieren Modelle mit synthetisch generierten Bewegungsmustern für Netzwerkumgebungen, 
für die noch keine Traces vorliegen. Dabei geht es darum, die Bewegung der Nutzer möglichst realistisch abzubilden,
wobei es verschiedene Arten von Ansätzen gibt, dies zu tun. Ein simpler Ansatz ist die vollständig zufällige Bewegungsmodellierung,
diese kann z.B. durch zeitliche oder räumliche Abhängigkeiten ergänzt werden. Auch geographischen Restriktionen können eine
Rolle spielen. Innerhalb der verschiedenen Arten von Ansätzen existiert eine Vielzahl von Mobilitätsmodellen, 
auf welche zum Teil im Verlauf dieser Ausarbeitung an geeigneter Stelle eingegangen wird.

\pagebreak

\subsection{Datenverkehr}
\label{sec:traffic}

Zu einer sinnvollen Simulation der Performanz eines Kommunikationssystems gehört auch die Simulation des Datenverkehrs
innerhalb dieses Systems. Wie bereits bei der Bewegungsmodellierung kann auch hier von tatsächlichen Beobachtungen in
existierenden Netzen ausgegangen werden, sofern solche vorliegen.
\newline\newline
Einfacher Datenverkehr besteht aus der einfachen Ankunft diskreter Entitäten (Pakete).
Es gibt allerdings auch Datenverkehr, bei welchem mehrere Entitäten zusammen in sogenannten \textquote{Batch-Arrivals} eintreffen.
Im einfachsten Fall wird in einer Simulation eine zufällige Sequenz von Zwischenankunftszeiten \footnote{Zeit zwischen zwei 
aufeinanderfolgenden \textquote{Arrivals}} generiert.
Neben Ankunftszeiten und Batch-Größen ist es häufig sinnvoll, die Last zu modellieren. \cite{Frost1994}
\newline\newline
Ein einfaches Modell des Datenverkehrs ist \textsc{CBR} (Constant Bitrate), bei welchem eine Anzahl von Paketankünften
pro Zeiteinheit definiert wird. Die Paketgröße ist dabei ebenfalls konstant. Dieses Modell ist sehr simpel und dementsprechend
leicht zu implementieren, dafür jedoch auch nicht besonders realistisch. Wenngleich VoIP und Video-Übertragung manchmal
die Gestalt von \textsc{CBR}-Datenverkehr aufweist. ADDCITE\newline
Neben \textsc{CBR}- gibt es auch \textsc{VBR}-Traffic (Variable Bitrate), welcher z.B. beim Videostreaming auftritt,
bei welchem kodierte Frames eine variable, zufällige Größe besitzen.\cite{Frost1994}
\newline\newline
Um einen realistischen, heterogenen Traffic-Mix zu erhalten, sollten verschiedene Ströme des Datenverkehrs
gemultiplext werden, z.B. Sprachverbindungen, Video-Übertragungen und Datei-Transporte. \cite{Frost1994}
\newline\newline
Eine weitere wichtige Eigenschaft des Datenverkehrs ist die sogenannte \textquote{Burstiness}, also die Stoßhaftigkeit.
Traffic entsteht in Schüben, z.B. für komprimierte Videos und Datentransfers. Die Ankünfte der Pakete bilden visuelle
Cluster. Der Datenverkehr in Breitbandnetzen wird von dieser Art Traffic dominiert. \cite{Frost1994}
\newline\newline
Wie bereits bei der Bewegungsmodellierung gibt es auch hier eine Vielzahl verschiedener Modelle, die in bestimmten Szenarien
geeignet sein können, um realistischen Datenverkehr zu modellieren.

\vfill

\pagebreak

\section{Korrelationen zwischen Mobilität und Datenverkehr}

Wie in den Abschnitten \ref{sec:mobility} und \ref{sec:traffic} deutlich wurde, hängt die Performanz eines mobilen
Netzwerks erheblich von den Mustern der Mobilität und des Datenverkehrs ab.
\newline\newline
Aktuelle Modelle betrachten entweder die Mobilität oder den Datenverkehr, erfassen jedoch nicht das Zusammenspiel beider Faktoren.
\cite{Alipour2018} Außerdem basieren viele trace-basierte Mobilitätsmodelle auf Datensätzen aus der Prä-Smartphone-Ära,
was die heutige Relevanz dieser Datensätze in Frage stellt. Die Autoren beschäftigen sich mit Mobilitäts- und Datenverkehrmodellen
für Laptops und Smartphones.
\newline\newline
Die Modelle sind datengetrieben, es werden $30$\textsc{TB} große Datensätze von $300 k$ Geräten betrachtet, welche auf einem Campusgelände
betrieben wurden. Zur Analyse dieser Daten wurde das Framework FLAMeS (\textbf{F}ramework for \textbf{L}arge-scale \textbf{A}nalysis of \textbf{M}obil\textbf{e} \textbf{S}ocieties) entwickelt. Das Ziel ist im Wesentlichen, einen ersten Schritt in Richtung integrierter
Mobilitäts-Datenverkehr-Modelle zu ermöglichen, um in der Zukunft realistischere Test-Szenarien und Benchmarks zu entwickeln.
\newline\newline
Die beiden Faktoren der Mobilität und des Datenverkehrs beeinflussen sich durchaus gegenseitig. Eine Person kann langsamer gehen,
wenn diese eine Nachricht erhält. Ebenso kann die Netzwerkaktivität von der Mobilität und Position abhängen. Stationäre Nutzer
konsumieren und produzieren i.d.R. mehr Daten als jene, die sich bewegen. Außerdem verwenden Personen an unterschiedlichen
Orten undterschiedliche Anwendungen und Services. Trotz dieses Zusammenhangs wurden Modelle zur Mobilität und zum Datenverkehr
bisher typischerweise isoliert voneinander betrachtet.
\newline\newline
Um diesen Zusammenhang genauer zu untersuchen, wird zwischen stationären Geräten (Laptops) und mobilen Geräten (Smartphones)
unterschieden. Das bereits erwähnte Framework FLAMeS wurde entwickelt, um stationäre und mobile Geräte zu differenzieren
und in Bezug auf ihren Datenverkehr und ihre Mobilität zu untersuchen.
\newline\newline
Im Kern wurden dafür drei Fragestellungen untersucht:\newline
\begin{itemize}
    \item Wie unterscheiden sich Mobilität und Datenverkehr zwischen den unterschiedlichen Geräteklassen, Zeit und Ort?
    \item Was sind die Beziehungen dieser Charakteristiken?
    \item Sollten neue Modelle entwickelt werden, die diese Unterschiede berücksichtigen? Wenn ja, wie?
\end{itemize}
\pagebreak
In Abb. \ref{fig:flames} ist eine Übersicht des FLAMeS-Systems dargestellt, welches aus drei Phasen besteht.
In der ersten Phase findet die Datensammlung und ein Preprocessing statt. Anschließend kommt es in Phase 2
zur Mobilitäts- und Traffic-Analyse, wobei zwischen Laptops und Smartphones unterschieden wird. In der letzten
Phase findet schließlich eine integrierte Analyse der Mobilität und des Traffics statt.

\begin{figure}[H]
\centering
\includegraphics[width=0.65\textwidth]{img/FLAMeS.png}
\caption{FLAMeS System. \cite{Alipour2018}}
\label{fig:flames}
\end{figure}

\subsection{I. Datensammlung und Preprocessing}

Die erste Quelle sind WLAN AP Logs. Diese Logs wurden von $1760$ APs in $138$ Gebäuden über $479$ Tage auf dem
Campus einer Universität gesammelt. Die Daten stammen von insgesamt $316k$ Geräten aus den Jahren $2011$ und $2012$.
Jeder Eintrag enthält die MAC-Adresse des Geräts, dessen zugewiesene IP-Adresse, den assoziierten AP und einen Zeitstempel.
Die Orte der APs werden durch die Längen- und Breitengrade der Gebäude, in welchen sie sich befinden, durch die Google Maps API
approximiert. Daneben gibt es als weitere Quelle NetFlow Logs. Aus demselben Netzwerk wurden im April $2012$ über $76$ Mrd.
Einträge in NetFlow Traces gesammelt. Ein \textquote{Flow} entspricht einer konsekutiven Sequenz von Paketen mit demselben
Transportprotokoll, Start- und Ziel-IP und Port.
\newline\newline
Anschließend werden die NetFlow-Einträge durch das dynamische MAC-to-IP Mapping aus den DHCP-Logs mit den WLAN-Verbindungen
aus den AP-Logs gematcht. Diese werden durch Ort und Webseiten-Informationen erweitert (durch rDNS).
\newline\newline
Danach geht es darum, die Geräte in Laptops und Smartphones zu klassifizieren. Das geschieht mithilfe
mehrerer Beobachtungen und Heuristiken. Zunächst einmal kann der Hersteller des Geräts mit der OUI
\footnote{Organizationally Unique Identifier - 24-Bit-Zahl, die eindeutig den Hersteller identifiziert} basierend auf den
ersten drei Oktetten der MAC-Adresse identifiziert werden. Auf diese Weise konnten bereits $46 \%$ der Geräte klassifiziert werden.
Die folgende Heuristik wird verwendet: Rufe sämtliche OUIs (MAC Prefix) ab, welche admob.com kontaktieren und markiere das Gerät
als Mobilgerät, wenn es bisher nicht markiert ist. Auf diese Weise konnten $86 \%$ der Geräte in AP-Logs und $97 \%$
der NetFlow-Traces klassifiziert werden. Wie man Abb. \ref{fig:traces} entnehmen kann, ist die Anzahl der Flows und der
Gesamt-Traffic bei Laptops deutlich höher als bei Mobilgeräten, obwohl es insgesamt deutlich mehr verbundene Mobilgeräte gibt.
Wie zu erwarten treten die Peaks an Wochentagen auf. Dass die Netzwerkaktivität nach dem 25. abnimmt, liegt daran, dass dort
die vorlesungsfreie Zeit beginnt.

\begin{figure}[H]
    \centering
    \includegraphics[width=0.65\textwidth]{img/traces.png}
    \caption{Traces über $25$ Tage. \cite{Alipour2018}}
    \label{fig:traces}
\end{figure}

\subsection{Mobilitätsanalyse}

In diesem Abschnitt geht es um eine zeitliche und örtliche Mobilitätsanalyse.
\newline\newline
Eine Session ist die Periode zwischen WLAN-Verbindungen. Die Startzeiten von Sessions stimmen mit den periodischen Startzeiten 
der Vorlesungen überein, allerdings hauptsächlich in Hörsälen. In diesen Gebäuden fällt die Aktivität von Laptops nach dem Ende
der Vorlesungszeiten stark. Die Aktivität von Mobilgeräten bleibt etwas länger erhalten. Für die Bibliothek und soziale
Einrichtungen auf dem Campus ist die Wahrscheinlichkeit für neue Sessions später am Abend deutlich höher. Diese klaren 
Ergebnisse gelten nur für Wochentage.
\newline\newline
Die räumliche Ausbreitung eines Geräts über einen Zeitraum von 6 Monaten.
Nach einer initialen Phase von ca. einer Woche stabilisiert sich diese Ausbreitung. Für Laptops
findet eine substanzielle Reduktion der Gesamtmobilität statt, während diese bei Smartphones nicht
in einer solchen Deutlichkeit vorliegt.
Da Smartphones \textquote{Always-On-Devices} sind, ist es leichter, deren Mobilität zu erfassen, da sie an Orten
wie Bushaltestellen etc., an denen Laptops typischerweise nicht eingeschaltet sind, verbunden sind.
Dies ermöglicht eine feingranularere Erfassung der Mobilität dieser Geräte.
Aufgrund der Vorlesungen, haben Stundenten, die an diesen teilnehmen an Wochentagen einen eingeschränkteren Bewegungsradius.
\newline\newline
Ebenfalls von Interesse ist die Zeit einer Session in einem Gebäude.
Laptops haben etwas längere Aufenthaltszeiten, im Median liegen allerdings beide Geräteklassen bei etwa $2:40$ Stunden.
Laptops werden in der Regel verwendet, wenn Nutzer länger an einem Ort verweilen.
\newline\newline
Trotz der geschilderten Unterschiede gibt es auch viele Gemeinsamkeiten, beispielsweise in der Verwendung während der Vorlesungszeit.

\subsection{Trafficanalyse}

In diesem Abschnitt geht es um eine Analyse des Datenverkehrs über unterschiedliche Gerätetypen, Zeiten und Orte.

Ein relevantes Kriterium ist die \textquote{Flow-Size}, also die Summe der Bytes aller Pakete eines Flows.
An Wochentagen ist die durchschnittliche Größe von Flows bei Smartphones mehr als doppelt so groß wie die von Laptop-Flows.
Auch an Wochenenden ändert sich dies nicht signifikant.
Die durchschnittliche Paketgröße von Smartphone-Flows ist $50 \%$ größer als die von Laptops.
Die Median-Größe der Pakete von Smartphones fällt an Wochenenden, während es bei Laptops gleich bleibt.
Laptop-Flows zeigen zwischen Wochentagen und Wochenden in Bezug auf die durchschnittliche Paketgröße keine signifikante Veränderung.
\newline\newline
Ebenfalls interessant ist die Laufzeit eines Flows. Beide Gerätetypen zeigen erhöhte Mittelwerte an Wochenenden.
Da es zusätzlich weniger aktive Geräte an Wochenenden gibt, zeigen die aktiven Geräte eine erhöhte Aktivität.
\newline\newline
Smartphones zeigen mehr extreme Phasen der Inaktivität, die durch größere Mobilität und Paketverluste verursacht werden können.
\newline\newline
Auch die Protokolle sind von Interesse. TCP ist für $78.5 \%$ der Laptop Flows verantwortlich
und für $98.2 \%$ der Smartphone Flows. Die höhere Präsenz von UDP bei Laptops ist nachvollziehbar, weil
UDP-Verbindungen wie Multiplayer-Spiele, Video-Konferenzen und Filesharing typischerweise nicht auf Mobilgeräten stattfinden.
\newline\newline
Insgesamt lässt sich festhalten, dass Smartphones mobiler sind und eine größere Anzahl von APs besuchen
und größere Flow-Sizes haben und trotzdem für weniger Netzwerklast verantwortlich sind.
\newline\newline
Auch das Verhalten der Nutzer ist relevant. An Wochentagen konsumieren $90 \%$ der Laptops weniger als $700$\textsc{MB},
während $90 \%$ der Smartphones weniger als $200$\textsc{MB} verbrauchen.
Bei der Paketrate sieht es ähnlich aus. An Wochentagen generieren Laptops durchschnittlich $318k$ Pakete, während
Smartphones nur durchschnittlich $84k$ Pakete am Tag generieren. An Wochenenden erhöhen sich die wenigen verbleibenden
Laptops deutlich und die Smartphones nur leicht.
\newline\newline
Die \textquote{Active Duration} ist weiterhin gut geeignet um Unterschiede zwischen Laptops und Smartphones zu verdeutlichen.
Dabei werden die NetFlows herangezogen und nicht die WLAN-AP Verbindungszeiten, denn dies ermöglicht es \textquote{Idle}-Zeiten
von aktiven Zeiten zu unterscheiden. Laptops haben 4x höhere durchschnittliche Aktivitätszeiten verglichen mit Smartphones.
Insgesamt sind über $90 \%$ der Laptops weniger als $3.5$ Stunden am Tag aktiv und $90 \%$ der Smartphones weniger als eine Stunde.
\newline\newline
Insgesamt scheint es, als wäre der Datanverbrauch von Smartphones mehr bursty mit größeren Flows und kleinerer aktiver Dauer.
Außerdem ist es so, dass an Wochenenden weniger Geräte am Campus sind, aber jene die verbleiben überdurchschnittlich aktiv sind
und überdurchschnittlich viel Daten konsumieren.
\newline\newline
Zuletzt wurden Machine-Learning-Verfahren angewandt, um ein gemischtes Modell zu trainieren, welches Datenpunkte
generiert und es konnte gezeigt werden, dass ein kombiniertes Mobilitäts-Traffic-Modell die Unterschiede besser abgebildet
als separierte Traffic- bzw. Mobilitäts-Modelle.
\newline\newline
Insgesamt wurde gezeigt, dass es ein signifikantes Potenzial für integrierte Mobilitäts- und Datenverkehr-Modelle gibt,
welche die Unterschiede und Beziehungen zwischen unterschiedlichen Gerätetypen, Zeiten und Orten abbilden.

\subsection{Diskussion}

TODO: Kernfragen diskutieren.

\pagebreak

\section{Fazit und Ausblick}

Die Autoren der in Abschnitt \ref{sec:} vorgestellten Veröffentlichung haben eine erweiterte Version veröffentlicht,
in welcher sie tiefer auf bestimmte zuvor betrachtete Aspekte eingehen...

%Bibliographie
\addcontentsline{toc}{section}{Literatur}
\bibliographystyle{IEEEtranSA}
%\bibliographystyle{acm}
\bibliography{sources.bib}

\end{document}
