\documentclass[12pt, a4paper]{article}
\usepackage{a4wide}

\usepackage[utf8]{inputenc}
\usepackage[ngerman]{babel}
\usepackage[T1]{fontenc}
\usepackage{palatino} %font

\usepackage{graphicx}
\usepackage{caption}
\usepackage{subcaption} %für subfigures
\usepackage{url}
\usepackage{tocloft}
\usepackage{acronym}
\usepackage{float}
\usepackage{color} 

\usepackage[babel,german=quotes,threshold=3]{csquotes} 

\usepackage{lipsum}
\usepackage{hyperref}  %hyperref still needs to be put at the end!

%Pfad für Grafiken
\graphicspath{{img/}}

%Styleregeln
\widowpenalty10000 % Vermeidet einzelne Zeilen eines Absatzes zu Beginn einer Seite
\clubpenalty10000 % Vermeidet einzelne Zeilen eines Absatzes am Ende einer Seite
\addtocontents{toc}{\protect\sloppy}
\setcounter{tocdepth}{3}

\begin{document}

%deaktiviere Seitenzahlen
\pagenumbering{gobble}

%Titelseite
\begin{titlepage}
\centering
\thispagestyle{empty}
\begin{center}
\includegraphics[width=0.9\textwidth]{uos.pdf}
\end{center}
\LARGE{\textsc{Institut für Informatik\\Arbeitsgruppe Verteilte Systeme}}
\vfill
\LARGE{\emph{Seminar}}\\
\LARGE{\emph{Mobility and Traffic in Computer Networks}}\\
\vspace{8mm}
\huge{\textbf{{\fontfamily{ppl}\selectfont
Analyzing Mobility-Traffic Correlations in Large WLAN Traces}}}\\
\vspace{9mm}
\LARGE{Tim Bohne}\\
\vspace{0.2cm}
%ACHTUNG: !!!Matrikelnummer nur für die Abgabeversion, NICHT mit ins Wiki hochladen!!!
% \normalsize{Matrikelnummer}\\
\vspace{4cm}
\large{Sommersemester 2019}\\
\vspace{0.2cm}
\large{\today}
\vfill
\end{titlepage}
\newpage

%Inhaltsverzeichnis
\tableofcontents
\newpage

\pagestyle{plain}
\pagenumbering{arabic} %Starte Seitennummerierung

\section{Einleitung}

Die stetig wachsende Zahl vernetzter Geräte führt zu enormen Herausforderungen bei der Entwicklung
und Planung der dafür erforderlichen Infrastruktur. Beim Konzipieren neuer Netzwerke ist es von zentraler
Bedeutung, diese in möglichst realistischer Weise zu simulieren, um sinnvolle Designentscheidungen
treffen zu können. Dabei spielen zwei Bereiche eine essenzielle Rolle, die Bewegung der Nutzer bzw. Geräte
sowie der Datenverkehr innerhalb des Netzwerks.
Diese Ausarbeitung befasst sich mit Korrelationen zwischen diesen Bereichen,
welche primär anhand der Ergebnisse des Papers \textquote{Flutes vs. Cellos: Analyzing Mobility-Traffic
Correlations in Large WLAN Traces} \cite{Alipour2018} aus dem Jahr 2018 erörtert werden.
Ziel des Papers ist es im Wesentlichen, einen ersten Schritt in Richtung integrierter Mobilitäts- und Datenverkehr-Modelle 
zu ermöglichen, um in Zukunft realistischere Test-Szenarien und Benchmarks zu entwickeln.
Darüber hinaus werden an einigen Stellen weiterführende Quellen einbezogen, um den Themenüberblick zu ergänzen.\newline
Zunächst werden in Kapitel \ref{sec:basics} die Grundlagen der Mobilität und des Datenverkehrs in Rechnernetzen
eingeführt. Anschließend geht es in Kapitel \ref{sec:correlations} um Korrelationen zwischen beiden
Bereichen, die anhand der Ergebnisse aus \cite{Alipour2018} vorgestellt werden.
Darauffolgend thematisiert Kapitel \ref{sec:mobility_predictability} die Vorhersagbarkeit menschlicher Mobilität,
bevor schließlich in Kapitel \ref{sec:conclusion} ein Fazit und ein Ausblick formuliert wird.

\section{Grundlagen}
\label{sec:basics}

Bevor Korrelationen zwischen Datenverkehr und Mobilität im WLAN sinnvoll thematisiert werden können, 
werden in diesem Kapitel zunächst die Grundlagen beider Konzepte eingeführt.
In Abschnitt \ref{sec:mobility} geht es um Mobilität in Drahtlosnetzwerken und Abschnitt \ref{sec:traffic} handelt
vom Datenverkehr in ebensolchen.

\subsection{Mobilität}
\label{sec:mobility}

Die Performanz eines kabellosen Kommunikationsnetzwerks hängt unter anderem von der Bewegung 
der Nutzer bzw. Geräte innerhalb dieses Netzwerks ab. Dementsprechend ist es erstrebenswert,
z.B. bei der Planung neuer Netze Simulationen mit möglichst realistischen Bewegungsmustern durchzuführen.
Ein Mobilitätsmodell beschreibt das Bewegungsmuster mobiler Benutzer bzw. Geräte,
also die Art, in welcher sich deren Standort, Geschwindigkeit und Beschleunigung im Laufe der Zeit ändern. \cite{Camp2002}
In \cite{Camp2002} wird dabei zwischen zwei grundsätzlichen Modellen unterschieden, den \textquote{Entity-Mobility-Models},
also Modellen, bei denen die Bewegung einzelner Entitäten modelliert wird und den \textquote{Group-Mobility-Models},
bei welchen es darum geht, die Bewegung einer Gruppe von Nutzern bzw. Geräten zu modellieren,
in welcher die Bewegung der einzelnen Entitäten voneinander abhängt.
Neben der Unterscheidung zwischen Entity- und Group-Mobility-Modellen, wird dort außerdem zwischen zwei Arten der Datenbasis unterschieden,
welche der Simulation zugrunde liegt. Es gibt zum einen die Bewegungsmodelle, die auf Trace-Daten basieren, d.h. auf Beobachtungen
der Bewegung von Nutzern in tatsächlich existierenden Systemen. Diese liefern bei einer ausreichend großen Gruppe von Nutzern
und einer ausreichend langen Beobachtungsphase sehr akurate Informationen, setzen allerdings voraus, dass ein solches Netzwerk
bereits existiert. \cite{Camp2002} Zum anderen existieren Modelle mit synthetischen Bewegungsmustern für Netzwerkumgebungen, 
für die noch keine Traces vorliegen. In diesen Fällen geht es darum, die Bewegung der Nutzer möglichst realistisch abzubilden,
wofür es verschiedene Arten von Ansätzen gibt.\newline
Wie in \cite{Aschenbruck2011} geschildert, können synthetische Modelle bestimmte Abhängigkeiten modellieren.
Zeitliche Abhängigkeiten sorgen z.B. dafür, dass die Bewegung einer Entität durch dessen Bewegung in der Vergangenheit
beeinflusst wird. Außerdem können räumliche Abhängigkeiten dafür sorgen, dass die Bewegung einer Entität durch die umgebenden 
Entitäten beeinflusst wird (Group-Mobility). Auch geographische Restriktionen können Teil des Modells sein.
Des Weiteren wird in \cite{Aschenbruck2011} darauf hingewiesen, dass auch Modelle wie
\textquote{Random-Waypoint} und \textquote{Random-Walk} existieren, welche keine dieser Abhängigkeiten
aufweisen. Diese Modelle sind einfach zu implementieren, jedoch nicht sehr realistisch, wenn es um die Modellierung
menschlicher Mobilität geht. Insgesamt existiert eine Vielzahl von Mobilitätsmodellen mit verschiedenen Ansätzen, 
welche in unterschiedlichen Szenarien geeignet sein können.

\subsection{Datenverkehr}
\label{sec:traffic}

Um den Anforderungen des mobilen Datenverkehrs gerecht zu werden und sinnvolle Strategien zu entwickeln,
ist es von zentraler Bedeutung, diese Anforderungen zu kennen, auch im Hinblick auf die Skalierbarkeit der
Netzwerkressourcen. \cite{Oliveira2014} Dementsprechend gehört zur Planung und Evaluation eines mobilen 
Kommunikationssystems auch die Simulation des Datenverkehrs innerhalb dieses Systems. Wie bereits bei der
Bewegungsmodellierung, kann auch hier von tatsächlichen 
Beobachtungen in existierenden Netzen ausgegangen werden, sofern solche vorliegen.\newline
Einfacher Datenverkehr besteht aus einzelnen Ankünften diskreter Entitäten.
Bei diesen Entitäten kann es sich z.B. um Pakete handeln. Darüber hinaus gibt es Datenverkehr, 
bei welchem mehrere Entitäten zusammen in sogenannten \textquote{Batch-Arrivals} eintreffen. \cite{Frost1994}
Die Grundlagen der Simulation des Datenverkehrs lassen sich basierend auf \cite{Frost1994} wie folgt zusammenfassen.
Im einfachsten Fall wird in einer Simulation eine zufällige Sequenz von Zwischenankunftszeiten \footnote{Zeit zwischen zwei 
aufeinanderfolgenden Paketankünften.} generiert. Neben Ankunftszeiten und Batch-Größen ist es zudem häufig sinnvoll,
die Last zu modellieren. Die Last entspricht der Menge der Arbeit $W_n$ im System, die die $n$-te eintreffende Entität 
erfordert.
% Ein einfaches Modell des Datenverkehrs ist \textsc{CBR} (Constant Bitrate), bei welchem eine Anzahl von Paketankünften
% pro Zeiteinheit definiert wird. Die Paketgröße ist dabei ebenfalls konstant. Dieses Modell ist sehr simpel und dementsprechend
% leicht zu implementieren, dafür jedoch in der Regel nicht besonders realistisch.\newline
% Neben \textsc{CBR}- gibt es den typischerweise realistischeren \textsc{VBR}-Traffic (Variable Bitrate), 
% welcher z.B. beim Videostreaming auftritt, bei dem kodierte Frames eine variable, zufällige Größe besitzen.\cite{Frost1994}
Um einen realistischen, heterogenen Traffic-Mix zu erhalten, sollten in einem Modell verschiedene
Ströme des Datenverkehrs gemultiplext werden, z.B. Sprachverbindungen, Video-Übertragungen und Datei-Transporte.
\newline\newline
Eine weitere wichtige Eigenschaft des Datenverkehrs ist die sogenannte \textquote{Burstiness}, also die Stoßhaftigkeit.
Datenverkehr entsteht typischerweise in Schüben, z.B. für komprimierte Videos und Datentransfers. Die Ankünfte der Pakete
bilden visuelle Cluster. Der Datenverkehr in Breitbandnetzen wird von dieser Art des Datenverkehrs dominiert. \cite{Frost1994}
Wie bereits bei der Bewegungsmodellierung, gibt es auch bei der Modellierung des Datenverkehrs eine Vielzahl verschiedener
Modelle, die in bestimmten Szenarien geeignet sein können, um realistischen Datenverkehr zu modellieren.

\section{Korrelationen zwischen Mobilität und Datenverkehr}
\label{sec:correlations}

Wie in den Abschnitten \ref{sec:mobility} und \ref{sec:traffic} eingeführt, hängt die Performanz eines
Netzwerks erheblich von den Mustern der Mobilität und des Datenverkehrs innerhalb dieses Netzwerks ab.
Es gibt in der Literatur zahlreiche Veröffentlichungen, die konkrete Modelle der Mobilität oder des Datenverkehrs
vorstellen. Verglichen mit diesen Veröffentlichungen handelt es sich bei \cite{Alipour2018} eher um eine Art 
\textquote{Meta-Paper}, in welchem keine konkreten Modelle vorgestellt werden. Stattdessen wird untersucht, 
ob kombinierte Modelle, die sowohl die Mobilität, als auch den Datenverkehr berücksichtigen, reale Netzwerke 
treffender repräsentieren als isolierte Varianten. Dabei werden Korrelationen zwischen beiden Faktoren untersucht.\newline
Aktuelle Modelle betrachten entweder die Mobilität oder den Datenverkehr, erfassen jedoch nicht das Zusammenspiel beider Faktoren.
Außerdem verwenden viele Trace-basierte Mobilitätsmodelle Datensätze aus der Prä-Smartphone-Ära,
was die heutige Relevanz dieser Modelle in Frage stellt. \cite{Alipour2018}
Die Autoren von \cite{Alipour2018} beschäftigen sich mit Mobilitäts- und Datenverkehrmodellen für Laptops und Smartphones.
Die Modelle sind datengetrieben, es werden $30$\textsc{TB} große Datensätze von $300$\textsc{K} Geräten betrachtet, 
welche auf einem Campusgelände betrieben wurden. Zur Analyse dieser Daten wurde das Framework FLAMeS \footnote{\textbf{F}ramework for 
\textbf{L}arge-scale \textbf{A}nalysis of \textbf{M}obil\textbf{e} \textbf{S}ocieties.} entwickelt.
Im Folgenden werden relevante Aspekte und Ergebnisse aus der Veröffentlichung zusammengefasst.\newline\newline
Dass die beiden Faktoren der Mobilität und des Datenverkehrs sich trotz der bisher überwiegend isolierten Betrachtungen
durchaus gegenseitig beeinflussen, veranschaulichen die Autoren durch folgende alltägliche Beispiele. 
Eine Person verlangsamt ihren Gang, wenn diese eine Nachricht erhält. Ebenso kann die Netzwerkaktivität von der
Mobilität und Position abhängen. Stationäre Nutzer konsumieren und produzieren in der Regel mehr Daten als jene, 
die sich bewegen. Außerdem verwenden Personen an unterschiedlichen Orten unterschiedliche Anwendungen.
Diese profanen Beispiele lassen eine gemeinsame Betrachtung beider Faktoren bereits als sinnvoll erscheinen.
Um diesen Zusammenhang genauer zu untersuchen, unterscheiden sie zwischen stationären Geräten (Laptops) und mobilen Geräten
(Smartphones). Da Laptops auch durchaus als mobil angesehen werden können, präzisieren die Autoren die Unterscheidung
durch \textquote{on-the-go}- und \textquote{stop-to-use}-Geräte. Das Framework FLAMeS ermöglicht es, 
in diesem Sinne stationäre und mobile Geräte zu differenzieren und in Bezug auf ihren Datenverkehr und ihre 
Mobilität zu untersuchen. Im Kern betrachten die Autoren dabei drei Fragestellungen:

\begin{itemize}
    \item Wie unterscheiden sich Mobilitäts- und Datenverkehr-Charakteristiken zwischen den unterschiedlichen Gerätetypen,
    Zeiten und Orten?
    \item Wie stehen diese Charakteristiken zueinander in Beziehung?
    \item Sollten neue Modelle entwickelt werden, die diese Unterschiede berücksichtigen?
\end{itemize}

Das FLAMeS-Systems besteht aus drei Phasen, in der ersten Phase findet die Datensammlung und ein Preprocessing statt
(vgl. \ref{sec:phase1}). Anschließend kommt es in der zweiten Phase zur Mobilitäts- und Datenverkehr-Analyse, wobei zwischen
Laptops und Smartphones unterschieden wird (vgl. \ref{sec:phase2_a}, \ref{sec:phase2_b}).
In der letzten Phase findet schließlich eine integrierte Analyse der Mobilität und des Datenverkehrs statt (vgl. \ref{sec:phase3}).
% \begin{figure}[H]
% \centering
% \includegraphics[width=0.65\textwidth]{img/FLAMeS.png}
% \caption{FLAMeS-System. \cite{Alipour2018}}
% \label{fig:flames}
% \end{figure}

\subsection{Datensammlung und Preprocessing}
\label{sec:phase1}

Die gesammelten Daten stammen aus zwei Quellen. Bei der ersten Quelle handelt es sich um WLAN-AP \footnote{Access-Point}-Logs,
welche an $1760$ APs in $138$ Gebäuden über $479$ Tage auf dem Campus einer Universität aufgezeichnet wurden. 
Die Daten stammen von insgesamt $316$\textsc{K} Geräten aus den Jahren $2011$ und $2012$. Jeder Eintrag enthält die 
MAC-Adresse des Geräts, dessen zugewiesene IP-Adresse, den AP, mit welchem das Gerät verbunden ist und einen Zeitstempel.
Die Orte der APs werden durch die Längen- und Breitengrade der Gebäude, in welchen sie sich befinden, mithilfe der Google Maps API
approximiert. Als weitere Quelle dienen Netflow \footnote{Ermöglicht Exportieren von Informationen über den IP-Datenstrom eines 
Netzwerks. \cite{RFC3954}}-Logs, die im selben Netzwerk im April $2012$ aus über $76$ Mrd.
Einträgen in Netflow-Traces generiert wurden. Ein \textquote{Flow} entspricht einer konsekutiven Sequenz von Paketen
mit demselben Transportprotokoll, Start- und Ziel-IP und Port. \cite{Alipour2018}
Die Netflow-Records werden durch das dynamische MAC-to-IP-Mapping mit den WLAN-Verbindungen aus den AP-Logs gematcht
und anschließend werden die Ergebnisse mittels rDNS \footnote{Reverse DNS} durch Ort und Webseiten-Informationen
ergänzt. Nachdem die Datenbasis geschaffen ist, geht es darum, die Geräte in Laptops und Smartphones zu klassifizieren.\newline
Zunächst kann der Hersteller des Geräts mit der OUI \footnote{Organizationally Unique Identifier - 24-Bit-Zahl, 
die eindeutig den Hersteller identifiziert.} basierend auf den ersten drei Oktetten der MAC-Adresse identifiziert werden.
Da viele Hersteller lediglich einen Gerätetyp herstellen, können auf diese Weise bereits
$46 \%$ der Geräte klassifiziert werden. Die Heuristik, welche von den Autoren zur Klassifizierung verwendet wird,
prüft zusätzlich, ob ein Gerät \textit{admob.com} kontatktiert, eine sehr verbreitete Plattform für Werbung auf Mobilgeräten.
Ist dies der Fall, so wird dieses Gerät als Mobilgerät klassifiziert. Auf diese Weise konnten $86 \%$ der Geräte in den AP-Logs
und $97 \%$ der Netflow-Traces klassifiziert werden. 
Die Autoren heben besonders hervor, dass diese Art der Geräteklassifizierung im Gegensatz zur Analyse von 
HTTP-Headern, die möglicherweise zu noch genaueren Ergebnissen führt, die Privatsphäre der Nutzer wahrt.\newline
Wie man Abb. \ref{fig:traces} entnehmen kann, ist die Anzahl der Flows und das Gesamtvolumen des Datenverkehrs bei Laptops
deutlich höher als bei Mobilgeräten, obwohl es insgesamt deutlich mehr verbundene Mobilgeräte gibt. Wie zu erwarten, 
treten die Spitzen an Wochentagen auf. Dass die Netzwerkaktivität nach dem 25. abnimmt, liegt daran, dass dort
die vorlesungsfreie Zeit beginnt.

\begin{figure}[H]
    \centering
    \includegraphics[width=0.65\textwidth]{img/traces.png}
    \caption{Kombinierte WLAN-AP- und Netflow-Traces. \cite{Alipour2018}}
    \label{fig:traces}
\end{figure}

\subsection{Mobilitätsanalyse}
\label{sec:phase2_a}

In diesem Abschnitt werden die Ergebnisse der Mobilitätsanalyse aus \cite{Alipour2018} zusammengefasst, in denen es
um die zeitliche und räumliche Analyse der Mobilität der verschiedenen Geräteklassen geht.
Die Startzeiten von WLAN-Sessions stimmen in Gebäuden, in denen Vorlesungen stattfinden mit den Startzeiten der Vorlesungen überein.
In diesen Gebäuden fällt die Aktivität von Laptops nach dem Ende der Vorlesungszeiten stark, während die Aktivität von Mobilgeräten
noch etwas länger erhalten bleibt. Für die Bibliothek und soziale Einrichtungen auf dem Campus ist die Wahrscheinlichkeit
für neue Sessions später am Abend höher. Diese Ergebnisse treffen nur auf Wochentage zu, deren Ablauf durch
die Vorlesungszeiten eine gewisse Struktur erhält.
Dies hat außerdem zur Folge, dass Stundenten, die an Vorlesungen teilnehmen, an Wochentagen einen eingeschränkteren Bewegungsradius haben.
Die Autoren haben die räumliche Ausbreitung der Geräte über einen Zeitraum von $6$ Monaten analysiert.
Nach einer initialen Phase von ca. einer Woche stabilisiert sich diese Ausbreitung. Für Laptops findet eine 
substanzielle Reduktion der Gesamtmobilität statt, während diese bei Smartphones nicht in einer solchen Deutlichkeit vorliegt.
Da Smartphones \textquote{always-on}-Geräte sind, ist es leichter, deren Mobilität zu erfassen, da sie auch an Orten
wie Bushaltestellen etc., an denen Laptops typischerweise nicht eingeschaltet sind, verbunden sind.
Dies ermöglicht eine genauere Erfassung der Mobilität dieser Geräte.
Außerdem wird die Anzahl der unterschiedlichen besuchten Gebäude eines Nutzers gezählt und ein bevorzugtes
Gebäude dieses Nutzers als jenes definiert, in welchem dieser an einem Tag die meiste Zeit verbracht hat.
Dazu werden die Session-Zeiten innerhalb eines Gebäudes aufsummiert.
Laptops haben etwas längere Aufenthaltszeiten und werden in der Regel verwendet, wenn Nutzer länger an einem Ort verweilen.

\subsection{Trafficanalyse}
\label{sec:phase2_b}

In diesem Abschnitt geht es um die zeitliche und räumliche Analyse des Datenverkehrs der verschiedenen Geräteklassen.
Im Folgenden werden die Ergebnisse der Traffic-Analyse von \cite{Alipour2018} vorgestellt.
Eine relevante Charakteristik ist die \textquote{Flow-Size}, also die Summe der Bytes aller Pakete eines Flows.
An Wochentagen ist die durchschnittliche Größe von Flows bei Smartphones mehr als doppelt so groß wie jene von Laptop-Flows.
Auch an Wochenenden ändert sich dies nicht signifikant. Die durchschnittliche Paketgröße von Smartphone-Flows
ist ca. $50 \%$ größer als die von Laptops. Die Median-Größe der Pakete von Smartphones fällt an Wochenenden, 
während diese bei Laptops gleich bleibt. Laptop-Flows zeigen zwischen Wochentagen und Wochenden in Bezug auf
die durchschnittliche Paketgröße keine signifikante Veränderung.
Trotz kleinerer Flows generiert ein durchschnittlicher Laptop $2.7$ mal so viel Traffic wie ein durchschnittliches Smartphone,
weil ein Laptop für $3.7$ mal so viele Flows verantwortlich ist, wie ein Smartphone.
\newline\newline
Ebenfalls interessant ist die Laufzeit eines Flows. Beide Gerätetypen zeigen erhöhte Mittelwerte an Wochenenden.
Da es zusätzlich weniger aktive Geräte an Wochenenden gibt, ist für die verbleibenden aktiven Geräte eine erhöhte Aktivität
zu beobachten. Smartphones zeigen insgesamt mehr extreme Phasen der Inaktivität, die unter anderem durch größere
Mobilität und Paketverluste verursacht werden können.
Auch die verwendeten Protokolle sind von Interesse. TCP ist für $78.5 \%$ der Laptop-Flows verantwortlich
und für $98.2 \%$ der Smartphone-Flows. Die höhere Präsenz von UDP bei Laptops ist zu erwarten, weil
UDP-Verbindungen, die z.B. bei Multiplayer-Spielen, Video-Konferenzen und Filesharing auftreten,
typischerweise weniger auf Mobilgeräten stattfinden.
Außerdem wurde die Last an den APs in sämtlichen Gebäuden täglich beobachtet. Dabei wurde darauf geachtet,
die Beobachtungen außerhalb der Klausurenphase zu tätigen, um Veränderungen im Verhalten der Nutzer zu vermeiden.
Es wurde die tägliche Rate der Paket- und Flowankünfte an den APs betrachtet. Die Median Flowraten sind
$42$\textsc{K} bzw. $20$\textsc{K} für Laptops und Smartphones an Wochentagen ($7.5$\textsc{K} bzw. $0.5$\textsc{K}
an Wochenenden). Die durchschnittliche Anzahl von Laptop-Paketen, die täglich von APs verarbeitet werden,
ist $1.6$ mal größer als die von Smartphones. An Wochenenden wird ein großer Anteil der APs nicht verwendet,
was die Aussage einer geringeren Mobilität der Nutzer an Wochenenden unterstützt.
Das tatsächliche Traffic-Volumen der APs beträgt an durchschnittlichen Wochentagen für $90 \%$ der APs
weniger als $5$\textsc{GB} Laptop-Traffic ($2.5$\textsc{GB} an Wochenenden) und weniger als $3$\textsc{GB} Smartphone-Traffic
($1$\textsc{GB} an Wochenenden).
Auch das Verhalten der Nutzer ist relevant. An Wochentagen \textcolor{red}{konsumieren} $90 \%$ der Laptops weniger als $700$\textsc{MB},
während $90 \%$ der Smartphones weniger als $200$\textsc{MB} verbrauchen.
Ein ähnliches Bild ergibt sich bei der Paketrate. An Wochentagen generieren Laptops durchschnittlich $318$\textsc{K} Pakete, 
während Smartphones nur durchschnittlich $84$\textsc{K} Pakete am Tag generieren. 
An Wochenenden erhöht sich die Paketrate und der Datenkonsum der wenigen verbleibenden Laptops deutlich,
während bei den Smartphones nur leichte Veränderungen zu beobachten sind.
Weiterhin gut geeignet, um Unterschiede zwischen Laptops und Smartphones zu verdeutlichen, sind die tatsächlichen Aktivitätszeiten.
Dabei werden die Netflows herangezogen und nicht die WLAN-AP-Verbindungszeiten, denn dies ermöglicht es \textquote{Idle}-Zeiten
von aktiven Zeiten zu unterscheiden. Laptops haben verglichen mit Smartphones $4$ mal höhere durchschnittliche Aktivitätszeiten.
Insgesamt sind über $90 \%$ der Laptops weniger als $3.5$ Stunden am Tag aktiv und $90 \%$ der Smartphones weniger als eine Stunde.\newline
Zusammenfassend lässt sich sagen, dass der Datanverbrauch von Smartphones mehr bursty mit größeren Flows und kleinerer aktiver
Dauer ist. Außerdem sind an Wochenenden weniger Geräte am Campus, aber jene die verbleiben, sind überdurchschnittlich aktiv 
und konsumieren überdurchschnittlich viele Daten. Insgesamt ist festzuhalten, dass Smartphones mobiler sind, 
eine größere Anzahl von APs besuchen, größere Flow-Sizes haben und trotzdem für insgesamt weniger Netzwerklast 
verantwortlich sind.

\subsection{Exkurs: Feature-Engineering}
Die Identifizierung relevanter Features ist essenziell, um Data-Mining-Algorithmen effektiv mit realen Daten zu verwenden.
Es wurden in der Literatur viele Feature-Selection Methoden vorgestellt, um Teilmengen der Features zu erhalten,
die für die Klassifizierung und Clustering relevant sind.
Machine-Learning-Methoden haben Schwierigkeiten, mit großen Zahlen von Input-Features
umzugehen. Um diese Methoden dennoch effektiv anzuwenden, ist ein Preprocessing der Daten vonnöten. \cite{Kumar2014}
Dieser Preprocessing-Schritt geschieht in der integrierten Mobilitäts- und Traffic-Analyse in \cite{Alipour2018} in
Form von Feature-Selection. Feature-Selection ist der Prozess, in welchem relevante Features von irrelevanten unterschieden
werden. Die irrelevanten Features werden nicht betrachtet, was die Genauigkeit, Geschwindigkeit und Verständlichkeit 
der Methoden verbessert. Irrelevante Features sind jene, die keine nützlichen Informationen liefern und redundante Features
liefern keine zusätzlichen Informationen. \cite{Kumar2014} Die betrachteten Features sollten dementsprechend weder
irrelevant noch redundant sein.

\subsection{Integrierte Mobilitäts- und Traffic-Modelle}
\label{sec:phase3}

In diesem Abschnitt geht es um den eigentlichen Kern von \cite{Alipour2018}, nämlich der Untersuchung, ob es einen Bedarf
für integrierte Mobility-Traffic-Modelle gibt. Um die Analyse und Interpretation zu vereinfachen und die 
wichtigsten Eigenschaften zu identifizieren, wird mithilfe eines \textsc{CFS}\footnote{Correlation Feature Selection}-Algorithmus
\textquote{Feature-Engineering} betrieben. \textsc{CFS} ist eine Möglichkeit der Feature-Selection, die auf Korrelationen
basiert. Die zentrale Hypothese des Ansatzes ist, dass gute Teilmengen der Features jene Features enthalten, die stark
mit der Klasse, jedoch nicht untereinander korrelieren. \textsc{CFS} kombiniert diese Hypothese mit einem geeigneten
Maß der Korrelation und einer heuristischen Suche. \cite{Hall2000}
Bei den in \cite{Alipour2018} ermittelten Korrelationen handelt es sich um \textquote{Pearson-Korrelationen}.
Die \textquote{Pearson-Correlation}-Methode vergibt Werte zwischen $-1$ und $1$, wobei $0$ überhaupt keiner Korrelation
entspricht. Bei den Werten $1$ und $-1$ handelt es sich um totale positive bzw. negative Korrelationen.
Eine positive Korrelation zwischen zwei Features $A$ und $B$ liegt vor, wenn bei steigendem Wert für $A$
auch der Wert von $B$ steigt. Wohingegen bei einer negativen Korrelation bei einer Steigerung des Werts von $A$, 
der Wert von $B$ fällt. \cite{Nettleton2014}
\newline\newline
\textbf{Mobilität}\newline
Der \textsc{CFS}-Algorithmus wird mit $8$ Mobilitäts-Eigenschaften ausgeführt und selektiert $5$ Eigenschaften, 
die für eine kombinierte Analyse geeignet sind. Für Laptops existiert z.B. an Wochentagen eine starke Korrelation ($0.96$)
zwischen der Zeit, die im bevorzugten Gebäude verbracht wird und der Dauer der Session, jedoch nur eine schwache Korrelation
($0.1$) an Wochenenden, was darauf hindeutet, dass die meiste Onlinezeit am Wochenende im bevorzugten
Gebäude verbracht wird. \cite{Alipour2018} Die vollständigen Ergebnisse zu Korrelationen der Mobilität kann in
Abb. \ref{fig:} nachvollzogen werden.
\newline\newline
\textbf{Traffic}\newline
Beim Traffic wird der \textsc{CFS}-Algorithmus mit $19$ Eigenschaften ausgeführt, von denen $11$
selektiert werden. Auch beim Traffic existieren einige Korrelationen, die in Abb. \ref{fig:} nachvollzogen werden können.
Die wohl interessanteste Erkenntnis ist allerdings, dass die Aktivitätszeiten und die Anzahl von Flows und Paketen nur
schwach korreliert, was bedeutet, dass Nutzer, die länger online sind, nicht notwendigerweise mehr Traffic erzeugen. \cite{Alipour2018}
\newline\newline
Ein wichtiger Schritt in Richtung integrierter Mobility-Traffic-Models ist die Analyse der Korrelationen
zwischen beiden Dimensionen (Mobilität und Traffic) basierend auf der Teilmenge der Eigenschaften,
die vom \textsc{CFS}-Algorithmus selektiert wurden.\newline
Die Ergebnisse lassen sich wie folgt zusammenfassen. Smartphones haben erwartungsgemäß hohe Scores bei den Mobilitätsmetriken (Bewegungsradius,
Anzahl besuchter unterschiedlicher Gebäude, ...), besitzen eine
insgesamt kleinere Anzahl von Flows und weniger Netzwerktraffic, aber produzieren im Durchschnitt größere Flows.
Für Laptops gilt an Wochenenden, je mehr Zeit in bevorzugten Gebäuden verbracht wird, desto größer die Gesamtaktivitätszeit
und Flow-Anzahl. Dieser Effekt existiert in abgeschwächter Form auch bei Smartphones. 
An Wochentagen existieren diese Korrelationen nicht.\newline
Die exakten Ergebnisse können in Abb. \ref{fig:} nachvollzogen werden.
\newline\newline
Mittels Machine Learning wurde untersucht, wie sich Mobilitäts- und Traffic-Features zwischen
Smartphones und Laptops unterscheiden. Anschließend wurde nach natürlichen konvexen Clustern von Usern im
Datensatz gesucht. Diese Schritte verifizieren, dass die Unterschiede der Mobilitäts- und Traffic-Charakteristiken
zwischen den Gerätetypen signifikant sind.
\newline\newline
\textbf{Supervised Classification}\newline
Die Autoren haben Support Vector Maschinen (SVM) auf verschiedenen Teilmengen von Features
verwendet, um die Zulässigkeit der Gerätetyp-Inferenz und die Beziehung zwischen Mobilitäts- und Traffic-Charakteristiken
zu untersuchen. Insgesamt wurde die Genauigkeit verschiedener Modelle analysiert. Dabei hat sich herausgestellt,
dass kombinierte Modelle, die sowohl Traffic als auch Mobilität berücksichtigen, mit $\approx 81 \%$ genauer sind als isolierte Modelle
(Mobilität $\approx 65 \%$, Traffic $\approx 79 \%$). Außerdem erhöht sich die Genauigkeit auf $\approx 86 \%$, wenn
zudem zwischen Wochentagen und Wochenenden differenziert wird.
\newline\newline
\textbf{Unsupervised Clustering}\newline
Um natürliche konvexe Cluster zu untersuchen, wurde ein $k$-Means-Algorithmus verwendet.
Wenn nur Mobilitäts-Features verwendet werden, werden ca. $60 \%$ der Geräte korrekt geclustert.
Werden dagegen nur Traffic-Features genutzt, erreicht man eine Genauigkeit von $81.2 \%$.
Werden Mobilität und Traffic kombiniert, erhöht sich die Genauigkeit leicht auf $81.5 \%$.
\newline\newline
\textbf{Mixture Model}\newline
Um einen Schritt in Richtung der Synthetisierung von Traces basierend auf den Datensätzen zu machen,
wurden Gaussian-Mixture-Models (GMM) mit kombinierten Mobilitäts- und Traffic-Features trainiert.
Anschlißend wurde Kolmogorov-Smirnov (KS) Statistik verwendet, um die generierten Samples mit Echtdaten zu vergleichen.
Dabei hat sich herausgestellt, dass das kombinierte Modell in der Lage ist, das Verhalten beider Gerätetypen abzubilden.
Das kombinierte Modell produziert Samples, dessen Traffic-Features den Originaldaten besser entsprechen, als ein GMM,
welches nur mit Traffic-Features trainiert wurde, was darauf hindeutet, dass eine Beziehung zwischen Mobilität und Traffic besteht.
Allerdings bringt das GMM des kombinierten Modells in Bezug auf Mobilitäts-Features verglichen mit dem Modell, welches 
ausschließlich mit Mobility-Features trainiert wurde, keine Verbesserung.
\newline\newline
Insgesamt sehen die Autoren ein signifikantes Potenzial integrierter Mobilitäts- und Traffic-Modelle, welche die Unterschiede
und Beziehungen von Features zwischen Gerätetypen, Zeit und Ort besser abbilden können als isolierte Modelle.

\subsection{Zusammenfassung}

In \cite{Alipour2018} wurde die Frage gestellt, ob es sinnvoll ist, Modelle zu entwickeln, die sowohl
Mobilitäts-, als auch Datenverkehrcharakteristiken berücksichtigen. Es wurde darin gezeigt, dass solche
kombinierten Betrachtungen potenziell zu deutlich realistischeren Ergebnissen gelangen können,
als die in der Regel zu simplen isolierten Betrachtungen.
Es wurde ein Framework entwickelt, um die Datensätze eines bestehenden Netzwerks dahingehend zu analysieren.
Zunächst ging es dabei um Geräteklassifizierung und anschließend um eine Untersuchung der Mobilitäts-
und Traffic-Metriken, wobei die verschiedenen Gerätetypen über Raum und Zeit hinweg miteinander verglichen wurden.
Nachdem gezeigt wurde, dass es signifikante Unterschiede zwischen den Gerätetypen gibt,
wurde mittels Machine-Learning ein gemischtes Modell trainiert, welches Datenpunkte generiert
und es konnte gezeigt werden, dass kombinierte Mobilitäts- und Traffic-Features die Unterschiede in den
Metriken besser abbilden, als isolierte Modelle. Die eingangs erwähnten Fragestellungen wurden beantwortet
und da viele der Erkenntnisse von aktuellen Modellen nicht berücksichtigt werden, liefert
das Paper wichtige Grundlagen für das Design zukünftiger, integrierter Modelle.

\section{Vorhersagbarkeit menschlicher Mobilität}
\label{sec:mobility_predictability}
Die Relevanz von Simulationen der Mobilität von Geräten bzw. Benutzern innerhalb eines Netzwerks
ist bereits ausführlich diskutiert worden. Um möglichst akurate Simulationen durchzuführen, wäre es erstrebenswert,
die Mobilität von realen Benutzern möglichst genau vorherzusagen. Mit dieser Vorhersagbarkeit haben sich die Autoren
von \cite{Alipour2018} in einer weiteren Veröffentlichung mit dem Titel \textquote{Practical Prediction of Human
Movements Across Device Types and Spatiotemporal Granularities}\cite{Alipour2019} aus diesem Jahr beschäftigt.
Sie kommen in der Arbeit zu dem Schluss, dass die Bewegung von Laptops oder allgemein \textquote{sit-to-use}-Geräten
signifikant besser vorherzusagen ist als jene von Smartphones bzw. Mobilgeräten.\newline
Sie haben signifikante Korrelationen zwischen Vorhersagegenauigkeit, Mobilitäts-Features
und Traffic-Features gefunden. Sie planen, den Nutzen der Vorhersagbarkeit als Feature 
in integrierten Mobilitäts- und Traffic-Modellen weiter zu untersuchen.

\section{Fazit und Ausblick}
\label{sec:conclusion}

Da ein Modell eine Repräsentation bestimmter Aspekte eines real existierenden Konzepts ist,
ist es von zentraler Bedeutung, diese Aspekte mit Bedacht zu wählen, sodass möglichst nur
Aspekte des Konzepts modelliert werden, die tatsächlich wichtig sind. Gleichzeitig sollten
sämtliche wichtigen Aspekte Teil des Modells sein.
Übertragen auf Modelle von Mobilfunknetzwerken, sind die bisher existierenden Modelle
basierend auf den Erkenntnissen aus \cite{Alipour2018} nicht detailliert genug. Es werden entweder
Traffic- oder Mobilitätscharakteristiken modelliert, jedoch nicht die Kombination beider Dimensionen.
Die betrachtete Veröffentlichung liefert demnach einen wichtigen Anstoß zur Entwicklung besserer Modelle.
\newline\newline
Die Autoren der in Abschnitt \ref{sec:} vorgestellten Veröffentlichung haben eine erweiterte Version veröffentlicht,
in welcher sie tiefer auf bestimmte zuvor betrachtete Aspekte eingehen und weitere Erkenntnisse in Bezug
auf das Design und die Parametrisierung von Mobilitäts- und Traffic-Modellen eingehen.
Die konkrete Konzeption und Validierung eines konkreten Modelles überlassen sie zukünftigen Arbeiten.
Es ist entscheidend, Laptops und Smartphones in Bezug auf Mobilität und Traffic zu differenzieren,
weil sie erhebliche Unterschiede aufweisen. Smartphones sind kontinuierlich präsent, während Laptops
zwischen Ortswechseln ausgeschaltet sind. Auch Unterschiede in den Trafficmustern sorgen dafür, dass
eine Unterscheidung der Gerätetypen notwendig ist. Zusätzlich können die einzelnen Faktoren
pro Gerätetyp und pro Nutzergruppe unterschieden werden. Studenten verschiedener Studiengänge könnten z.B.
differenziert werden.
\newline\newline
Bereits das alltägliche Beispiel der Person, die ihren Gang verlangsamt um eine empfangene Nachricht zu lesen,
verdeutlicht, dass es einen Zusammengang zwischen Mobilität und Datenverkehr bei Mobilgeräten gibt.
In \cite{Alipour2018} wird diese intuitive Idee eines Zusammenhangs bestätigt. Da beide Bereiche
bisher typischerweise isoliert voneinander betrachtet werden, handelt es sich um ein aktives Forschungsfeld,
bei dem in Zukunft weitere Betrachtungen und konkretere Ansätze kombinierter Modelle zu erwarten sind.
\newline\newline
Die Autoren von \cite{Alipour2018} haben wichtige Erkenntnisse für den zukünftigen Entwurf von integrierten Modellen
geliefert, die für zahlreiche praktische Anwendungen von Relevanz sein können. Allerdings haben sie noch kein konkretes
Modell vorgestellt, dies bleibt Aufgabe zukünftiger Forschung. Zum gegenwärtigen Zeitpunkt liegt noch kein konkretes Modell
vor, weshalb es spannend bleibt, abzuwarten, wie groß die Verbesserungen, die sich mit den Vorschlägen aus \cite{Alipour2018}
praktisch erreichen lassen, ausfallen werden.

\vfill
\pagebreak

%Bibliographie
\addcontentsline{toc}{section}{Literatur}
\bibliographystyle{IEEEtranSA}
%\bibliographystyle{acm}
\bibliography{sources.bib}

\end{document}
